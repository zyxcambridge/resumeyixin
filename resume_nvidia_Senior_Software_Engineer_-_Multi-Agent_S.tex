
% ============================================
% 定制化简历 - 针对岗位
% 职位: Senior Software Engineer - Multi-Agent System - AV Infrastructure
% URL: https://nvidia.wd5.myworkdayjobs.com/en-US/NVIDIAExternalCareerSite/details/Senior-Software-Engineer--Multi-Agent-System---AV-Infrastructure_JR2010348
% 生成时间: 2026-01-01 17:39:18
% 
% 匹配关键词:
% - 技术栈: Multi-agent, Distributed Systems, AI Agent, Autonomous Vehicle, LLM, autonomous vehicle, CUDA, PyTorch
% - 领域: Autonomous, autonomous
% ============================================
\documentclass[11pt,a4paper]{article}
\usepackage[UTF8, scheme=plain]{ctex}
\usepackage{geometry}
\usepackage{graphicx}
\usepackage{hyperref}
\usepackage{fontawesome5}
\usepackage{enumitem}
\usepackage{xcolor}
\usepackage{titlesec}

% Chinese font setup - ctex handles this automatically
% Override if needed
\setCJKmainfont{Heiti SC}
\setmainfont{Times New Roman}

% Page geometry
\geometry{
    left=1.5cm,
    right=1.5cm,
    top=1.5cm,
    bottom=1.5cm
}

% Hyperref setup
\hypersetup{
    colorlinks=true,
    linkcolor=black,
    urlcolor=blue,
    citecolor=black
}

% Section formatting
\titleformat{\section}
{\large\bfseries}
{}
{0em}
{}[\titlerule]

\titlespacing*{\section}
{0pt}
{12pt}
{8pt}

% Remove page numbers
\pagestyle{empty}

\begin{document}

% Header with name
\begin{center}
    {\Large \textbf{张益新 (Yixin Zhang)}}
\end{center}
\vspace{8pt}

% Photo and contact info layout using tabular
% \noindent
% \begin{tabular}{@{}p{0.25\textwidth}@{\hspace{0.05\textwidth}}p{0.7\textwidth}@{}}
% \raisebox{-0.5\height}{\includegraphics[width=0.23\textwidth]{image/algo_yuanrong/1756452787751.png}} &
% \begin{tabular}[c]{@{}l@{}}
% \faEnvelope \hspace{2pt} \href{mailto:zyxcambridge@gmail.com}{zyxcambridge@gmail.com} \\[8pt]
% \faPhone \hspace{2pt} 17521398109 \\[8pt]
% \faMapMarker \hspace{2pt} 上海
% \end{tabular}
% \end{tabular}

% Contact info without photo
\begin{center}
\begin{tabular}[c]{@{}l@{}}
\faEnvelope \hspace{2pt} \href{mailto:zyxcambridge@gmail.com}{zyxcambridge@gmail.com} \\[8pt]
\faPhone \hspace{2pt} 17521398109 \\[8pt]
\faMapMarker \hspace{2pt} 上海
\end{tabular}
\end{center}

\vspace{12pt}

\begin{center}
    {\large \textbf{算法工程师}}
\end{center}

\vspace{8pt}
\hrule
\vspace{8pt}

\section*{工作经历}

\textbf{VLN部署算法工程师} | 某上市公司机器人子公司 | 2025.9 - 至今

\vspace{4pt}
\begin{itemize}[leftmargin=*,itemsep=2pt,parsep=1pt]
    \item \textbf{项目名称:} 视觉语言导航(VLN)算法部署与自动跟随机器人系统
    \item \textbf{核心算法能力:}
    \begin{itemize}[leftmargin=*,noitemsep]
        \item \textbf{长距离规划:} 支持超过150米的长距离导航,实现复杂环境下的路径规划
        \item \textbf{零样本泛化:} 实现陌生环境自主适配能力,无需预训练即可在新场景中稳定运行
        \item \textbf{密集障碍穿行:} 突破密集障碍物挑战,在障碍间距小于50cm的极端场景下仍可安全通过
        \item \textbf{动态避障:} 实时感知并避开移动障碍物,保障机器人安全运行
    \end{itemize}
    \item \textbf{模型部署与优化:}
    \begin{itemize}[leftmargin=*,noitemsep]
    \item 在NVIDIA Thor和Orin平台上成功部署VLN模型,实现端到端推理优化
    \item 通过模型量化、算子融合、内存优化等技术手段,显著降低时间延迟
    \item 完成自动跟随机器人功能开发,实现稳定可靠的跟随性能
    \item 成功完成42次关键测试验证,系统稳定性和可靠性达到量产标准
    \item 具身智能算法研究部署落地:在NVIDIA Thor和Orin平台上实现具身智能算法的工程化部署
    \end{itemize}
    \item \textbf{项目战果:}
    \begin{itemize}[leftmargin=*,noitemsep]
        \item 在几乎"零基础"的条件下,独立完成软硬件全链路打通,包括Thor硬件配置、算法视频调优等全部工作
        \item 成功实现跟随功能稳定运行,攻克了项目中最难啃的技术难题
        \item 建立了完整的部署流程和验证标准,为后续规模化应用奠定基础
    \end{itemize}
    \item \textbf{12.18演示救场:}
    \begin{itemize}[leftmargin=*,noitemsep]
        \item 在演示现场语音网络全面崩溃的紧急情况下,凭借Plan B预案和完整的控制环境备份
        \item 现场统筹协调团队,稳定团队心态,确保第二次演示顺利通过
        \item 在高压环境下突破心理阈值,展现了出色的项目把控和应急处理能力
    \end{itemize}
    \item \textbf{项目规划与团队建设:}
    \begin{itemize}[leftmargin=*,noitemsep]
        \item 在demo成功运行的基础上,制定Thor项目规模化扩展计划
        \item 申请团队扩充,规划基础工作分工,聚焦核心技术攻坚
        \item 从"单兵作战"向"集团军冲锋"转变,争取更大的技术突破和商业价值
    \end{itemize}
\end{itemize}

\textbf{智能体算法研发(竞赛期间)} | 自由职业 | 2025.2 - 2025.9

\vspace{4pt}
\begin{itemize}[leftmargin=*,itemsep=2pt,parsep=1pt]
    \item \textbf{竞赛成就:} NeurIPS 2025智能体工具增强推理Workshop - CureBench国际智能体评测竞赛全球第二名(Top 2)
    \item \textbf{论文发表:} 作为共同第一作者发表论文"CureAgent: A Training-Free Executor-Analyst Framework for Clinical Reasoning" (arXiv:2512.05576)
    \item \textbf{项目背景:} 参与NeurIPS 2025顶级会议Workshop,构建生物医学领域专业Agent系统。针对小规模LLM临床智能体的上下文利用失败问题,提出Executor-Analyst框架
    \item \textbf{技术架构:} CureBench + TxAgent RL Framework - ART Training
    \item \textbf{核心工作:}
    \begin{itemize}[leftmargin=*,noitemsep]
        \item 设计并实现Executor-Analyst模块化架构,将工具执行与临床推理解耦,缓解单体模型的推理缺陷
        \item 提出分层集成策略(Stratified Ensemble),通过保留证据多样性解决信息瓶颈问题
        \item 发现上下文-性能悖论和动作空间维度诅咒等关键缩放洞察
        \item 利用Agent和Test-time Scaling技术构建用药助手智能体
        \item 调用FDA、OpenTargets、PubMed等生物医学工具进行智能体工具增强推理
        \item 完成《自进化智能体–动态记忆与持续运行的架构实践》一书初稿撰写
    \end{itemize}
\end{itemize}

\textbf{深度学习算法工程师} | 安波福中央电气(上海)有限公司 | 2024.8 - 2025.2

\vspace{4pt}
\begin{itemize}[leftmargin=*,itemsep=2pt,parsep=1pt]
    \item \textbf{项目名称:} 通用障碍物感知算法研发与端到端量产预研
    \item \textbf{模型部署工作分层:}
    \begin{itemize}[leftmargin=*,noitemsep]
        \item 部署一个网络
        \item 部署多个网络,在一个芯片上进行性能优化:
        \begin{itemize}[leftmargin=*,noitemsep]
            \item 2个backbone + 3个head,多任务流水线
            \item 共享内存和队列,instance bank
        \end{itemize}
        \item 统一适配多个芯片框架(GPU+ASIC+FPGA):
        \begin{itemize}[leftmargin=*,noitemsep]
            \item 多芯片平台调度框架
            \item 一套代码适配多个芯片
        \end{itemize}
    \end{itemize}
    \item 参与L2++无图端到端网络设计方案,主导通用障碍物OCC分支的网络结构设计和优化
    \item 负责芯片选型,完成三种芯片的benchmark测试,设计端到端网络多任务调度框架
    \item 建立部署工作标准SOP和work flow,独立完成OD和OCC分支在2种芯片的AI模型部署
    \item \textbf{异步调度:} 设计异步流水线,实现CPU和多个AI加速核心(NPU/VP)最大限度重叠工作,缩短总体完成时间
    \item \textbf{奔向空闲:} 将1000ms任务压缩到28ms(14ms*2),使芯片更快进入低功耗idle状态
    \item \textbf{系统级优化 - 多任务调度框架:} 设计串行、并行或流水线并行的Pipeline编排多个模型执行顺序
    \begin{itemize}[leftmargin=*,noitemsep]
        \item 车间A (Stage 1): 特征提取中心,配备两台并行机器分别计算Backbone\_1 (BEV)和Backbone\_2 (Temporal)
        \item 车间B (Stage 2): 感知分析部,配备两台并行机器分别计算Head\_1 (OD)和Head\_2 (Map)
        \item 车间C (Stage 3): 决策融合站,负责最后的Head\_3 (Predict)计算
    \end{itemize}
    \item \textbf{部署与验证工作流 (量产规范):}
    \begin{itemize}[leftmargin=*,noitemsep]
        \item 模型导出与分段: 将PyTorch模型导出为ONNX并按DAG节点分割为独立文件(backbone1.onnx, head1.onnx等)
        \item 模型编译: 使用硬件厂商工具链将ONNX编译为二进制bin文件,完成算子融合与量化优化
        \item 分阶段验证:
        \begin{itemize}[leftmargin=*,noitemsep]
            \item IO对齐: 确保部署程序与仿真脚本输入逐比特一致
            \item 单节点验证: 对比bin文件输出与PC端ONNX结果
            \item 多帧端到端对齐: 验证完整APP业务指标(mAP, IoU)与黄金参考模型一致性
        \end{itemize}
        \item \textbf{最终集成与交付}: 封装为符合规范的5个核心API(Init/Run/Release/GetResult/GetStatus)
        \item 核心优化指标: 时间延迟、吞吐量、内存带宽(减少占用)、功耗(W4A4)、算力占用(软硬一体)
        \item 适配车机芯片: 英伟达Orin、地平线J5/J6、 CV3
    \end{itemize}
\end{itemize}

\textbf{模型部署负责人/高级软件架构} | 图达通智能科技(上海)有限公司 | 2022.6-2024.04

\vspace{4pt}
\begin{itemize}[leftmargin=*,itemsep=2pt,parsep=1pt]
    \item 招聘组建高效的深度学习部署团队,完善算法工程落地流程
    \item 基于Jetson平台实现AI LiDAR模型实时推理,时间延迟降低4倍
    \item 开发ADAS代码补全VSCode插件,提升自动驾驶算法开发效率
    \item 大模型推理部署探索,成功部署Ollama、llama.cpp、vllm、TensorRT-LLM、mlc-llm等框架
    \item 在Mac M1上成功运行yi-34B模型,在Android平台使用mlc-llm运行llama3-8b模型
    \item 基于NVIDIA Drive Sim完善MLOps流程,产生10W帧仿真数据集
    \item 实现模型仿真数据和真实数据的联合训练,精度提高10个百分点
    \item \textbf{【模型部署】}
    \begin{itemize}[leftmargin=*,noitemsep]
        \item 性能分析工具应用:使用Nsight Systems和Nsight Compute定位部署瓶颈,解决关键性能卡点
        \item lidar车端框架构建:实现激光雷达点云检测与语义分割网络部署迭代,包括rangeimage和pillar voxel网络实时推理,设计端到端统一检测分割框架
        \item NVIDIA技术协作:与Jetson团队建立直接沟通渠道,分析模型瓶颈并获取优化支持,通过任务依赖关系优化将整体延迟降低75\%(四分之一)
        \item 地平线生态对接:为地平线J5平台争取硬件和软件开发支持,完成感知算法全量移植
        \item 技术难题突破:与SPConv核心开发团队协作,解决自定义稀疏卷积算子的部署挑战
    \end{itemize}
\end{itemize}

\textbf{高级算法工程师} | 上海雪湖科技有限公司 | 2019.05-2022.05

\vspace{4pt}
\begin{itemize}[leftmargin=*,itemsep=2pt,parsep=1pt]
    \item 从0到1搭建算法团队,构建V2X感知算法工程化落地流程
    \item 设计FPGA混合量化方案,精细化到乘法和加法运算级别
    \item 成功量产MEC设备100台以上,软硬一体变现500多万,服务近10家客户
    \item \textbf{【难点与解决方案】}
    \begin{itemize}[leftmargin=*,noitemsep]
        \item 精度损失问题:针对量化导致的30个点精度损失,提出混合量化策略
        \item 量化方案:实现PTQ(Post-training Quantization),参考QAT(Quantization-aware Training)全流程
        \item 量化算法对比:系统评估minmax、KL散度和MSE等量化方法的误差分布特性
        \item 工具开发:自主研发精度比对工具,可视化不同量化参数下的误差热力图
        \item 训练推理一致性:从量化原理公式出发,解决训练/推理精度偏差问题
        \item 客户信任构建:通过理论推导和实测数据双重验证,解决时间延迟质疑
    \end{itemize}
    \item \textbf{【实现方法】}
    \begin{itemize}[leftmargin=*,noitemsep]
        \item 核心优化流程:构建哈希表$\to$生成Rulebook$\to$Gather输入特征成稠密矩阵M\_in$\to$执行GEMM$\to$Scatter输出结果
    \end{itemize}
    \item \textbf{【lidar MLOPS】}
    \begin{itemize}[leftmargin=*,noitemsep]
        \item 分层策略:关键特征提取层保留浮点精度,分类头和后处理层量化至INT8
        \item 量化算法:采用KL散度校准方法,使用校准集进行数据分布匹配
        \item 网络分解:将复杂卷积操作分解为基本乘加运算,适配FPGA整数计算架构
        \item 基于ZCU104 IP域控制器构建深度学习部署流程,实现10倍速度提升
    \end{itemize}
    \item \textbf{【技术落地】}
    \begin{itemize}[leftmargin=*,noitemsep]
        \item 根据项目和现有公司解决方案创建V2X解决方案文档,提供现场解释,获得10+客户
        \item 在项目现场进行现场开发,在中国各地城市实施10+试点项目
        \item 推动FPGA lidar加速IP向速腾、禾赛、Ouster、Livox等激光雷达公司的适配
    \end{itemize}
    \item \textbf{【项目收入证明】}
    \begin{itemize}[leftmargin=*,noitemsep]
        \item 基于AMD Xilinx ZCU104大规模生产100+MEC设备,实现200万元+软硬件综合收入
        \item 拥有近10个客户,建立长期合作关系
    \end{itemize}
\end{itemize}

\textbf{算法工程师} | 深兰科技(上海)有限公司 | 2018.05 - 2019.02

\vspace{4pt}
\begin{itemize}[leftmargin=*,itemsep=2pt,parsep=1pt]
    \item \textbf{汇报对象:} 技术经理
    \item \textbf{基于关键点的目标检测算法工程落地:}
    \begin{itemize}[leftmargin=*,noitemsep]
        \item 实现基于OpenPose的人体关键点检测算法,自主完成C++推理流程
        \item 算法改进包括高斯响应增强、Heatmap半径调整和中继监督机制
        \item 创新实现基于峰值的NMS方法替代传统IoU方式
    \end{itemize}
    \item \textbf{模型部署与优化:}
    \begin{itemize}[leftmargin=*,noitemsep]
        \item 在P100 GPU上部署模型,单卡支持28个模型并行推理
        \item 将OpenPose推理时延优化至500ms以内
    \end{itemize}
    \item \textbf{算法验证:}
    \begin{itemize}[leftmargin=*,noitemsep]
        \item 构建密集场景测试集,在复杂环境下达到99\%关键点识别准确率
    \end{itemize}
    \item \textbf{其他成果:}
    \begin{itemize}[leftmargin=*,noitemsep]
        \item 获得3项发明专利授权
        \item 开发基于关键点的云端商品识别服务
    \end{itemize}
\end{itemize}

\textbf{深度学习工程师} | 友邦网络科技有限公司 | 2017.12 - 2018.04

\vspace{4pt}
\begin{itemize}[leftmargin=*,itemsep=2pt,parsep=1pt]
    \item \textbf{汇报对象:} 技术经理
    \item \textbf{ADAS系统开发:}
    \begin{itemize}[leftmargin=*,noitemsep]
        \item 研发行人和车辆目标检测算法,结合图像分割技术实现车道线检测
        \item 在RK3399等嵌入式平台进行网络优化,显著提升检测速度
        \item 使用TensorFlow Lite进行Android端模型移植和量化优化
    \end{itemize}
    \item \textbf{人体姿态识别:}
    \begin{itemize}[leftmargin=*,noitemsep]
        \item 基于MobileNet架构实现轻量级人体关键点检测
        \item 分别使用Caffe2和TensorFlow框架完成算法实现
        \item 成功移植至Android设备,支持实时推理
    \end{itemize}
\end{itemize}


\vspace{10pt}
\rule{\textwidth}{1pt}
\vspace{8pt}

\section*{教育背景}

\textbf{网络工程学士} | 北华航天工业学院 | 2010.09-2014.06

\vspace{10pt}
\rule{\textwidth}{1pt}
\vspace{8pt}

\section*{核心成就与著作}

\textbf{论文发表:}
\begin{itemize}[leftmargin=*,itemsep=2pt,parsep=1pt]
    \item \textbf{CureAgent: A Training-Free Executor-Analyst Framework for Clinical Reasoning} (arXiv:2512.05576). Ting-Ting Xie, \textbf{Yixin Zhang}. NeurIPS 2025 Workshop - CURE-Bench Competition 2nd Place Solution. \href{https://arxiv.org/abs/2512.05576}{arXiv:2512.05576}
\end{itemize}

\textbf{著作出版:}
\begin{itemize}[leftmargin=*,itemsep=2pt,parsep=1pt]
    \item 撰写《自进化智能体–动态记忆与持续运行的架构实践》一书,目前已完成初稿并进入出版社三审三校阶段,计划由电子工业出版社出版
\end{itemize}

\textbf{竞赛成就:}
\begin{itemize}[leftmargin=*,itemsep=2pt,parsep=1pt]
    \item NeurIPS 2025智能体工具增强推理Workshop - CureBench国际智能体评测竞赛全球第二名(Top 2)
\end{itemize}

\vspace{10pt}
\rule{\textwidth}{1pt}
\vspace{8pt}

\section*{项目与成就}

\textbf{【项目名称:通用障碍物感知算法研发与端到端量产预研】}

\vspace{4pt}
\begin{itemize}[leftmargin=*,itemsep=2pt,parsep=1pt]
    \item \textbf{项目目标}:研发通用障碍物检测算法,基于物理规律(深度分布和语义约束)实现开放场景泛化感知,预研车规级低功耗平台(16W)端到端部署方案
    \item \textbf{核心技术突破}:
    \begin{itemize}[leftmargin=*,noitemsep]
        \item 范式转变:从'学习物体形状'转向'验证物理规律',通过深度分布建模与语义约束验证,实现'不识形状却感知存在'的泛化能力
        \item 场景优化:强化建筑区域垂直边缘梯度约束,降低悬挂物误识别风险
        \item 负障碍物检测:利用高度突变和点密度异常等几何特征,实现坑洼、沟渠等负障碍物识别
    \end{itemize}
    \item \textbf{核心挑战与解决方案}:
    \begin{itemize}[leftmargin=*,noitemsep]
        \item 挑战:端到端感知(OD+Map+Predict+Plan)计算量与16W功耗限制的矛盾
        \item 解决方案:
        \begin{itemize}[leftmargin=*,noitemsep]
            \item 1) 模型轻量化:优化OD与Map模型参数结构
            \item 2) 混合量化策略:FP16+INT8混合量化,结合离线初始化与在线校准,精度损失$<$2\%
            \item 3) 算子定制:开发体素化、可变形聚合等关键算子,提升计算效率
        \end{itemize}
    \end{itemize}
    \item \textbf{关键成果}:
    \begin{itemize}[leftmargin=*,noitemsep]
        \item 端到端推理速度提升10倍,成功部署至目标平台
        \item 混合量化方法申请技术专利
        \item 验证基于物理规律感知范式在开放场景的可行性
    \end{itemize}
    \item \textbf{前瞻性研究}:
    \begin{itemize}[leftmargin=*,noitemsep]
        \item 混合精度训练梯度不匹配问题解决方案
        \item 端侧内存带宽限制下的算子融合方案
    \end{itemize}
\end{itemize}

\textbf{【项目名称:lidar 推理框架】}

\vspace{4pt}
\begin{itemize}[leftmargin=*,itemsep=2pt,parsep=1pt]
    \item \textbf{难点与解决方案}
    \begin{itemize}[leftmargin=*,noitemsep]
        \item \textbf{精度损失控制}:针对量化导致的30个点精度损失,提出混合量化策略
        \item \textbf{量化方案实现}:搭建PTQ/QAT全流程,支持minmax、KL散度、MSE三种校准算法
        \item \textbf{误差分析工具}:自主研发精度比对平台,可视化不同量化参数下的误差热力图
        \item \textbf{训练推理一致性}:从量化原理公式出发,解决浮点/整数计算偏差问题
        \item \textbf{客户信任构建}:通过理论推导+实测数据双重验证,将延迟降低至客户要求的1/3
    \end{itemize}
    \item \textbf{技术实现}
    \begin{itemize}[leftmargin=*,noitemsep]
        \item 核心优化流程:构建哈希表$\to$生成Rulebook$\to$Gather输入特征$\to$执行GEMM$\to$Scatter输出结果
        \item 量化算法对比:系统评估三种方法在不同层的误差分布,选择最优组合策略
    \end{itemize}
    \item \textbf{商业价值}
    \begin{itemize}[leftmargin=*,noitemsep]
        \item 基于AMD Xilinx ZCU104大规模生产100+MEC设备
        \item 实现200万元+软硬件综合收入,服务近10家客户
        \item 量产陕重汽
    \end{itemize}
\end{itemize}

\textbf{【项目名称:NeurIPS 2025 CureBench医药Agent系统(CureAgent)】}

\vspace{4pt}
\begin{itemize}[leftmargin=*,itemsep=2pt,parsep=1pt]
    \item \textbf{项目角色:} NeurIPS 2025 Workshop项目负责人/核心开发者、论文共同第一作者
    \item \textbf{项目背景:} 参与NeurIPS 2025顶级会议Workshop,构建生物医学领域专业Agent系统。针对当前基于小规模LLM(如TxAgent)的临床智能体存在的\textbf{上下文利用失败(Context Utilization Failure)}问题,提出Executor-Analyst框架,将工具执行的语法精确性与临床推理的语义鲁棒性解耦。
    \item \textbf{技术架构:} CureBench + TxAgent RL Framework - ART Training
    \begin{itemize}[leftmargin=*,noitemsep]
        \item \textbf{Executor-Analyst框架:} 模块化架构,将专门的TxAgent执行器(Executors)与长上下文基础模型分析师(Analysts)协同工作,缓解单体模型的推理缺陷
        \item \textbf{分层集成策略(Stratified Ensemble):} 通过保留证据多样性,显著优于全局池化,有效解决信息瓶颈问题
        \item \textbf{训练无关架构工程:} 无需昂贵的端到端微调,在CURE-Bench上达到最先进性能
    \end{itemize}
    \item \textbf{核心技术突破:}
    \begin{itemize}[leftmargin=*,noitemsep]
        \item \textbf{上下文-性能悖论(Context-Performance Paradox):} 发现推理上下文超过12k tokens会引入噪声导致准确性下降
        \item \textbf{动作空间维度诅咒:} 工具集扩展需要分层检索策略
        \item 调用FDA、OpenTargets、PubMed等生物医学工具进行智能体工具增强推理
        \item 实现Test-time Scaling技术,提升智能体在测试阶段的性能
        \item 基于强化学习框架(TxAgent RL Framework)进行智能体训练与优化
    \end{itemize}
    \item \textbf{项目成果:}
    \begin{itemize}[leftmargin=*,noitemsep]
        \item 在CureBench国际智能体评测竞赛中获得全球第二名(Top 2)
        \item 发表论文至arXiv(arXiv:2512.05576),代码已开源
        \item 成功构建了可实际应用的生物医学领域智能体系统,为下一代可信AI驱动治疗学提供可扩展、敏捷的基础
    \end{itemize}
\end{itemize}

\textbf{大模型微调与部署项目}

\vspace{4pt}
\begin{itemize}[leftmargin=*,itemsep=2pt,parsep=1pt]
    \item 在AWS上部署Baichuan2模型为OpenAI兼容格式,实现Lora微调
    \item 使用LLaMA-Factory在阿里云平台进行大模型微调与优化
    \item 将Start Code模型部署至HuggingFace作为API服务
    \item 优化8B模型性能达到相当于671B模型水平
\end{itemize}

\vspace{10pt}
\rule{\textwidth}{1pt}
\vspace{8pt}

\section*{学术与竞赛经历}

\begin{itemize}[leftmargin=*,itemsep=2pt,parsep=1pt]
    \item 2025年: NeurIPS 2025智能体工具增强推理Workshop - CureBench国际智能体评测竞赛全球第二名(Top 2)
    \item 2021年10月: CVPR Workshop - 3D目标检测算法(第九名)
    \item 多次黑客马拉松获奖:
    \begin{itemize}[leftmargin=*,itemsep=1pt]
        \item 2023年10月: 百川黑客马拉松 - 美国中医获客Agent项目(特别奖)
        \item 2023年9月: Google I/O Hackathon - Stable Diffusion计算共享(三等奖)
        \item 2023年7月: 世界AI黑客马拉松 - 领导力追踪与训练(二等奖)
        \item 2019年9月: AngelHack上海 - 领导力追踪与训练(一等奖)
        \item 2017年7月: 全球AI黑客马拉松 - 假新闻检测(冠军)
    \end{itemize}
\end{itemize}

\vspace{10pt}
\rule{\textwidth}{1pt}
\vspace{8pt}

\section*{技术社区参与}

\begin{itemize}[leftmargin=*,itemsep=2pt,parsep=1pt]
    \item Google机器学习开发专家,连续5年每年举办4场技术讲座,影响11,874人
\end{itemize}

\end{document}

