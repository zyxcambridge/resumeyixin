
% ============================================
% 定制化简历 - AGI岗位申请
% 公司: Google
% 职位: Software Engineer - AI Agents (Agentic AI)
% URL: https://careers.google.com/jobs/results/?q=AI%20Agent
% 地点: Mountain View, CA / New York, NY
% 生成时间: 2026-01-04 16:16:06
% 
% 岗位关键词: AI Agent, Agentic AI, LLM, Reasoning, Planning, Tool Use
% 匹配分数: 2
% 匹配优势:
% - 
NeurIPS 2025 CureBench Global 2nd Place - AI Agent development% - CureAgent framework, Executor-Analyst architecture, tool-augmented reasoning% - Deployed Ollama, llama.cpp, vllm, TensorRT-LLM, mlc-llm
% ============================================
\documentclass[11pt,a4paper]{article}
\usepackage{fontspec}
\usepackage{geometry}
\usepackage{graphicx}
\usepackage{hyperref}
\usepackage{fontawesome5}
\usepackage{enumitem}
\usepackage{xcolor}
\usepackage{titlesec}

% Font setup
\setmainfont{Times New Roman}

% Page geometry
\geometry{
    left=1.5cm,
    right=1.5cm,
    top=1.5cm,
    bottom=1.5cm
}

% Hyperref setup
\hypersetup{
    colorlinks=true,
    linkcolor=black,
    urlcolor=blue,
    citecolor=black
}

% Section formatting
\titleformat{\section}
{\large\bfseries}
{}
{0em}
{}[\titlerule]

\titlespacing*{\section}
{0pt}
{12pt}
{8pt}

% Remove page numbers
\pagestyle{empty}

\begin{document}

% Header with name
\begin{center}
    {\Large \textbf{Yixin Zhang}}
\end{center}
\vspace{8pt}

% Contact info
\begin{center}
\begin{tabular}[c]{@{}l@{}}
\faEnvelope \hspace{2pt} \href{mailto:zyxcambridge@gmail.com}{zyxcambridge@gmail.com} \\[8pt]
\faPhone \hspace{2pt} +86 17521398109 \\[8pt]
\faMapMarker \hspace{2pt} Shanghai, China
\end{tabular}
\end{center}

\vspace{12pt}

\begin{center}
    {\large \textbf{Algorithm Engineer}}
\end{center}

\vspace{8pt}
\hrule
\vspace{8pt}

\section*{Work Experience}

\textbf{VLN Deployment Algorithm Engineer} | Robotics Subsidiary of a Listed Company | Sep. 2025 - Present

\vspace{4pt}
\begin{itemize}[leftmargin=*,itemsep=2pt,parsep=1pt]
    \item \textbf{Project:} Vision-Language Navigation (VLN) Algorithm Deployment and Autonomous Following Robot System
    \item \textbf{Core Algorithm Capabilities:}
    \begin{itemize}[leftmargin=*,noitemsep]
        \item \textbf{Long-range Planning:} Support navigation over 150 meters, enabling path planning in complex environments
        \item \textbf{Zero-shot Generalization:} Achieve autonomous adaptation in unfamiliar environments, stable operation in new scenarios without pre-training
        \item \textbf{Dense Obstacle Navigation:} Break through dense obstacle challenges, safely pass through extreme scenarios with obstacle spacing less than 50cm
        \item \textbf{Dynamic Obstacle Avoidance:} Real-time perception and avoidance of moving obstacles, ensuring safe robot operation
    \end{itemize}
    \item \textbf{Model Deployment \& Optimization:}
    \begin{itemize}[leftmargin=*,noitemsep]
        \item Successfully deployed VLN models on NVIDIA Thor and Orin platforms, achieving end-to-end inference optimization
        \item Significantly reduced latency through model quantization, operator fusion, memory optimization, and other technical means
        \item Completed autonomous following robot functionality development, achieving stable and reliable following performance
        \item Successfully completed 42 critical test verifications, system stability and reliability reached mass production standards
        \item Embodied Intelligence Algorithm Research \& Deployment: Achieved engineering deployment of embodied intelligence algorithms on NVIDIA Thor and Orin platforms
    \end{itemize}
    \item \textbf{Project Achievements:}
    \begin{itemize}[leftmargin=*,noitemsep]
        \item Independently completed full-stack hardware-software integration from almost "zero foundation", including Thor hardware configuration, algorithm video tuning, and all related work
        \item Successfully achieved stable following functionality, overcoming the most challenging technical difficulties in the project
        \item Established complete deployment processes and verification standards, laying the foundation for subsequent large-scale applications
    \end{itemize}
    \item \textbf{Dec. 18 Demo Rescue:}
    \begin{itemize}[leftmargin=*,noitemsep]
        \item In an emergency situation where the voice network completely crashed during the demo, leveraged Plan B and complete control environment backup
        \item Coordinated the team on-site, stabilized team morale, ensuring the second demo passed successfully
        \item Demonstrated excellent project control and emergency handling capabilities under high-pressure conditions
    \end{itemize}
    \item \textbf{Project Planning \& Team Building:}
    \begin{itemize}[leftmargin=*,noitemsep]
        \item Developed Thor project scale-up plan based on successful demo operation
        \item Applied for team expansion, planned basic work division, focused on core technology breakthroughs
        \item Transitioned from "solo operation" to "group army charge", pursuing greater technical breakthroughs and business value
    \end{itemize}
\end{itemize}

\textbf{Agent Algorithm R\&D (Competition Period)} | Freelance | Feb. 2025 - Sep. 2025

\vspace{4pt}
\begin{itemize}[leftmargin=*,itemsep=2pt,parsep=1pt]
    \item \textbf{Competition Achievement:} NeurIPS 2025 Agent Tool-Augmented Reasoning Workshop - CureBench International Agent Evaluation Competition Global 2nd Place (Top 2)
    \item \textbf{Publication:} Co-first author of paper "CureAgent: A Training-Free Executor-Analyst Framework for Clinical Reasoning" (arXiv:2512.05576)
    \item \textbf{Project Background:} Participated in NeurIPS 2025 top-tier conference Workshop, built professional Agent system in biomedical field. Addressed Context Utilization Failure in small LLM-based clinical agents by proposing Executor-Analyst Framework
    \item \textbf{Technical Architecture:} CureBench + TxAgent RL Framework - ART Training
    \item \textbf{Core Work:}
    \begin{itemize}[leftmargin=*,noitemsep]
        \item Designed and implemented Executor-Analyst modular architecture, decoupling tool execution from clinical reasoning, mitigating reasoning deficits in monolithic models
        \item Proposed Stratified Ensemble strategy, addressing information bottleneck by preserving evidentiary diversity
        \item Discovered critical scaling insights including Context-Performance Paradox and Curse of Dimensionality in action spaces
        \item Built medication assistant agent using Agent and Test-time Scaling technologies
        \item Integrated biomedical tools (FDA, OpenTargets, PubMed) for agent tool-augmented reasoning
        \item Completed initial draft of book "Self-Evolving Agents - Architecture Practice of Dynamic Memory and Continuous Operation"
    \end{itemize}
\end{itemize}

\textbf{Deep Learning Algorithm Engineer} | Aptiv Central Electrical (Shanghai) Co., Ltd. | Aug. 2024 - Feb. 2025

\vspace{4pt}
\begin{itemize}[leftmargin=*,itemsep=2pt,parsep=1pt]
    \item \textbf{Project:} General Obstacle Perception Algorithm R\&D and End-to-End Mass Production Pre-research
    \item \textbf{Model Deployment Work Hierarchy:}
    \begin{itemize}[leftmargin=*,noitemsep]
        \item Deploy a single network
        \item Deploy multiple networks, optimize performance on a single chip:
        \begin{itemize}[leftmargin=*,noitemsep]
            \item 2 backbones + 3 heads, multi-task pipeline
            \item Shared memory and queues, instance bank
        \end{itemize}
        \item Unified adaptation to multiple chip frameworks (GPU+ASIC+FPGA):
        \begin{itemize}[leftmargin=*,noitemsep]
            \item Multi-chip platform scheduling framework
            \item One codebase adapts to multiple chips
        \end{itemize}
    \end{itemize}
    \item Participated in L2++ mapless end-to-end network design, led network structure design and optimization of general obstacle OCC branch
    \item Responsible for chip selection, completed benchmark testing of three chip types, designed end-to-end network multi-task scheduling framework
    \item Established deployment work standards SOP and workflow, independently completed AI model deployment of OD and OCC branches on 2 chip types
    \item \textbf{Asynchronous Scheduling:} Designed asynchronous pipeline, maximizing overlap between CPU and multiple AI acceleration cores (NPU/VP), reducing total completion time
    \item \textbf{Race to Idle:} Compressed 1000ms tasks to 28ms (14ms*2), enabling chips to enter low-power idle state faster
    \item \textbf{System-level Optimization - Multi-task Scheduling Framework:} Designed Pipeline orchestrating multiple model execution sequences in serial, parallel, or pipelined parallel modes
    \begin{itemize}[leftmargin=*,noitemsep]
        \item Stage A (Stage 1): Feature extraction center, equipped with two parallel machines computing Backbone\_1 (BEV) and Backbone\_2 (Temporal) respectively
        \item Stage B (Stage 2): Perception analysis department, equipped with two parallel machines computing Head\_1 (OD) and Head\_2 (Map) respectively
        \item Stage C (Stage 3): Decision fusion station, responsible for final Head\_3 (Predict) computation
    \end{itemize}
    \item \textbf{Deployment \& Verification Workflow (Mass Production Standards):}
    \begin{itemize}[leftmargin=*,noitemsep]
        \item Model export \& segmentation: Export PyTorch models to ONNX and segment into independent files by DAG nodes (backbone1.onnx, head1.onnx, etc.)
        \item Model compilation: Use hardware vendor toolchains to compile ONNX to binary bin files, complete operator fusion and quantization optimization
        \item Phased verification:
        \begin{itemize}[leftmargin=*,noitemsep]
            \item IO alignment: Ensure bit-by-bit consistency between deployment program and simulation script inputs
            \item Single node verification: Compare bin file outputs with PC-side ONNX results
            \item Multi-frame end-to-end alignment: Verify complete APP business metrics (mAP, IoU) consistency with golden reference model
        \end{itemize}
        \item \textbf{Final Integration \& Delivery:} Encapsulated as 5 core APIs conforming to standards (Init/Run/Release/GetResult/GetStatus)
        \item Core optimization metrics: Time latency, throughput, memory bandwidth (reduced usage), power consumption (W4A4), compute utilization (software-hardware integration)
        \item Adapted to automotive chips: NVIDIA Orin, Horizon J5/J6, CV3
    \end{itemize}
\end{itemize}

\textbf{Model Deployment Lead / Senior Software Architect} | Innovusion Intelligent Technology (Shanghai) Co., Ltd. | Jun. 2022 - Apr. 2024

\vspace{4pt}
\begin{itemize}[leftmargin=*,itemsep=2pt,parsep=1pt]
    \item Recruited and built efficient deep learning deployment team, improved algorithm engineering implementation processes
    \item Achieved real-time AI LiDAR model inference on Jetson platform, reduced latency by 4x
    \item Developed ADAS code completion VSCode plugin, improved autonomous driving algorithm development efficiency
    \item Explored large model inference deployment, successfully deployed Ollama, llama.cpp, vllm, TensorRT-LLM, mlc-llm and other frameworks
    \item Successfully ran yi-34B model on Mac M1, ran llama3-8b model on Android platform using mlc-llm
    \item Improved MLOps process based on NVIDIA Drive Sim, generated 100K frame simulation dataset
    \item Achieved joint training of simulation and real data, improved accuracy by 10 percentage points
    \item \textbf{[Model Deployment]}
    \begin{itemize}[leftmargin=*,noitemsep]
        \item Performance analysis tool application: Used Nsight Systems and Nsight Compute to locate deployment bottlenecks, solved critical performance issues
        \item LiDAR vehicle-side framework construction: Achieved LiDAR point cloud detection and semantic segmentation network deployment iteration, including rangeimage and pillar voxel network real-time inference, designed end-to-end unified detection-segmentation framework
        \item NVIDIA technical collaboration: Established direct communication channel with Jetson team, analyzed model bottlenecks and obtained optimization support, reduced overall latency by 75\% (one quarter) through task dependency optimization
        \item Horizon ecosystem integration: Secured hardware and software development support for Horizon J5 platform, completed full perception algorithm migration
        \item Technical challenge breakthrough: Collaborated with SPConv core development team to solve custom sparse convolution operator deployment challenges
    \end{itemize}
\end{itemize}

\textbf{Senior Algorithm Engineer} | Shanghai Xuehu Technology Co., Ltd. | May 2019 - May 2022

\vspace{4pt}
\begin{itemize}[leftmargin=*,itemsep=2pt,parsep=1pt]
    \item Built algorithm team from scratch, constructed V2X perception algorithm engineering implementation process
    \item Designed FPGA hybrid quantization scheme, refined to multiplication and addition operation levels
    \item Successfully mass-produced 100+ MEC devices, achieved 5M+ RMB in software-hardware revenue, served nearly 10 customers
    \item \textbf{[Challenges \& Solutions]}
    \begin{itemize}[leftmargin=*,noitemsep]
        \item Accuracy loss issue: Proposed hybrid quantization strategy for 30-point accuracy loss caused by quantization
        \item Quantization scheme: Implemented PTQ (Post-training Quantization), referenced QAT (Quantization-aware Training) full process
        \item Quantization algorithm comparison: Systematically evaluated error distribution characteristics of minmax, KL divergence, and MSE quantization methods
        \item Tool development: Independently developed accuracy comparison tool, visualized error heatmaps under different quantization parameters
        \item Training-inference consistency: Solved training/inference accuracy deviation from quantization principle formulas
        \item Customer trust building: Resolved latency concerns through theoretical derivation and measured data dual verification
    \end{itemize}
    \item \textbf{[Implementation Method]}
    \begin{itemize}[leftmargin=*,noitemsep]
        \item Core optimization flow: Build hash table $\to$ Generate Rulebook $\to$ Gather input features into dense matrix M\_in $\to$ Execute GEMM $\to$ Scatter output results
    \end{itemize}
    \item \textbf{[LiDAR MLOPS]}
    \begin{itemize}[leftmargin=*,noitemsep]
        \item Layered strategy: Key feature extraction layers retain floating-point precision, classification head and post-processing layers quantized to INT8
        \item Quantization algorithm: Adopted KL divergence calibration method, used calibration set for data distribution matching
        \item Network decomposition: Decomposed complex convolution operations into basic multiply-add operations, adapted to FPGA integer computation architecture
        \item Built deep learning deployment process based on ZCU104 IP domain controller, achieved 10x speed improvement
    \end{itemize}
    \item \textbf{[Technical Implementation]}
    \begin{itemize}[leftmargin=*,noitemsep]
        \item Created V2X solution documents based on projects and existing company solutions, provided on-site explanations, acquired 10+ customers
        \item Conducted on-site development at project locations, implemented 10+ pilot projects in various cities across China
        \item Promoted FPGA LiDAR acceleration IP adaptation to RoboSense, Hesai, Ouster, Livox and other LiDAR companies
    \end{itemize}
    \item \textbf{[Project Revenue Proof]}
    \begin{itemize}[leftmargin=*,noitemsep]
        \item Mass-produced 100+ MEC devices based on AMD Xilinx ZCU104, achieved 2M+ RMB in software-hardware revenue
        \item Maintained nearly 10 customers with long-term cooperative relationships
    \end{itemize}
\end{itemize}

\textbf{Algorithm Engineer} | DeepBlue Technology (Shanghai) Co., Ltd. | May 2018 - Feb. 2019

\vspace{4pt}
\begin{itemize}[leftmargin=*,itemsep=2pt,parsep=1pt]
    \item \textbf{Reporting to:} Technical Manager
    \item \textbf{Keypoint-based Object Detection Algorithm Engineering:}
    \begin{itemize}[leftmargin=*,noitemsep]
        \item Implemented OpenPose-based human keypoint detection algorithm, independently completed C++ inference pipeline
        \item Algorithm improvements included Gaussian response enhancement, Heatmap radius adjustment, and intermediate supervision mechanism
        \item Innovatively implemented peak-based NMS method replacing traditional IoU approach
    \end{itemize}
    \item \textbf{Model Deployment \& Optimization:}
    \begin{itemize}[leftmargin=*,noitemsep]
        \item Deployed models on P100 GPU, single card supported 28 models for parallel inference
        \item Optimized OpenPose inference latency to within 500ms
    \end{itemize}
    \item \textbf{Algorithm Validation:}
    \begin{itemize}[leftmargin=*,noitemsep]
        \item Built dense scene test set, achieved 99\% keypoint recognition accuracy in complex environments
    \end{itemize}
    \item \textbf{Other Achievements:}
    \begin{itemize}[leftmargin=*,noitemsep]
        \item Obtained 3 invention patent authorizations
        \item Developed cloud-based product recognition service based on keypoints
    \end{itemize}
\end{itemize}

\textbf{Deep Learning Engineer} | Youbang Network Technology Co., Ltd. | Dec. 2017 - Apr. 2018

\vspace{4pt}
\begin{itemize}[leftmargin=*,itemsep=2pt,parsep=1pt]
    \item \textbf{Reporting to:} Technical Manager
    \item \textbf{ADAS System Development:}
    \begin{itemize}[leftmargin=*,noitemsep]
        \item Developed pedestrian and vehicle object detection algorithms, combined with image segmentation technology to achieve lane detection
        \item Optimized networks on embedded platforms such as RK3399, significantly improved detection speed
        \item Used TensorFlow Lite for Android model migration and quantization optimization
    \end{itemize}
    \item \textbf{Human Pose Recognition:}
    \begin{itemize}[leftmargin=*,noitemsep]
        \item Implemented lightweight human keypoint detection based on MobileNet architecture
        \item Completed algorithm implementation using Caffe2 and TensorFlow frameworks respectively
        \item Successfully ported to Android devices, supporting real-time inference
    \end{itemize}
\end{itemize}


\vspace{10pt}
\rule{\textwidth}{1pt}
\vspace{8pt}

\section*{Education}

\textbf{Bachelor of Network Engineering} | North China Institute of Aerospace Engineering | Sep. 2010 - Jun. 2014

\vspace{10pt}
\rule{\textwidth}{1pt}
\vspace{8pt}

\section*{Highlights \& Publications}

\textbf{Publications:}
\begin{itemize}[leftmargin=*,itemsep=2pt,parsep=1pt]
    \item \textbf{CureAgent: A Training-Free Executor-Analyst Framework for Clinical Reasoning} (arXiv:2512.05576). Ting-Ting Xie, \textbf{Yixin Zhang}. NeurIPS 2025 Workshop - CURE-Bench Competition 2nd Place Solution. \href{https://arxiv.org/abs/2512.05576}{arXiv:2512.05576}
\end{itemize}

\textbf{Book:}
\begin{itemize}[leftmargin=*,itemsep=2pt,parsep=1pt]
    \item Authored book "Self-Evolving Agents - Architecture Practice of Dynamic Memory and Continuous Operation", currently completed initial draft and entered publisher's three-review and three-proofreading stage, planned to be published by Publishing House of Electronics Industry
\end{itemize}

\textbf{Competition Achievement:}
\begin{itemize}[leftmargin=*,itemsep=2pt,parsep=1pt]
    \item NeurIPS 2025 Agent Tool-Augmented Reasoning Workshop - CureBench International Agent Evaluation Competition Global 2nd Place (Top 2)
\end{itemize}

\vspace{10pt}
\rule{\textwidth}{1pt}
\vspace{8pt}

\section*{Selected Projects}

\textbf{[Project: General Obstacle Perception Algorithm R\&D and End-to-End Mass Production Pre-research]}

\vspace{4pt}
\begin{itemize}[leftmargin=*,itemsep=2pt,parsep=1pt]
    \item \textbf{Project Goal:} Develop general obstacle detection algorithm, achieve open-scene generalized perception based on physical laws (depth distribution and semantic constraints), pre-research end-to-end deployment solution for automotive-grade low-power platform (16W)
    \item \textbf{Core Technical Breakthroughs:}
    \begin{itemize}[leftmargin=*,noitemsep]
        \item Paradigm shift: From 'learning object shapes' to 'verifying physical laws', achieving 'perceiving existence without recognizing shapes' generalization capability through depth distribution modeling and semantic constraint verification
        \item Scene optimization: Enhanced vertical edge gradient constraints in building areas, reduced false detection risk of suspended objects
        \item Negative obstacle detection: Utilized geometric features such as height mutations and point density anomalies to achieve detection of negative obstacles like potholes and ditches
    \end{itemize}
    \item \textbf{Core Challenges \& Solutions:}
    \begin{itemize}[leftmargin=*,noitemsep]
        \item Challenge: Contradiction between end-to-end perception (OD+Map+Predict+Plan) computation and 16W power consumption limit
        \item Solutions:
        \begin{itemize}[leftmargin=*,noitemsep]
            \item 1) Model lightweighting: Optimized OD and Map model parameter structures
            \item 2) Hybrid quantization strategy: FP16+INT8 hybrid quantization, combined offline initialization with online calibration, accuracy loss $<$2\%
            \item 3) Custom operators: Developed key operators such as voxelization and deformable aggregation, improved computational efficiency
        \end{itemize}
    \end{itemize}
    \item \textbf{Key Results:}
    \begin{itemize}[leftmargin=*,noitemsep]
        \item End-to-end inference speed improved 10x, successfully deployed to target platform
        \item Hybrid quantization method applied for technical patent
        \item Verified feasibility of physics-based perception paradigm in open scenes
    \end{itemize}
    \item \textbf{Forward-looking Research:}
    \begin{itemize}[leftmargin=*,noitemsep]
        \item Solutions for gradient mismatch issues in mixed precision training
        \item Operator fusion solutions under edge-side memory bandwidth constraints
    \end{itemize}
\end{itemize}

\textbf{[Project: LiDAR Inference Framework]}

\vspace{4pt}
\begin{itemize}[leftmargin=*,itemsep=2pt,parsep=1pt]
    \item \textbf{Challenges \& Solutions}
    \begin{itemize}[leftmargin=*,noitemsep]
        \item \textbf{Accuracy Loss Control:} Proposed hybrid quantization strategy for 30-point accuracy loss caused by quantization
        \item \textbf{Quantization Scheme Implementation:} Built PTQ/QAT full process, supported minmax, KL divergence, MSE three calibration algorithms
        \item \textbf{Error Analysis Tool:} Independently developed accuracy comparison platform, visualized error heatmaps under different quantization parameters
        \item \textbf{Training-Inference Consistency:} Solved floating-point/integer computation deviation from quantization principle formulas
        \item \textbf{Customer Trust Building:} Reduced latency to 1/3 of customer requirements through theoretical derivation + measured data dual verification
    \end{itemize}
    \item \textbf{Technical Implementation}
    \begin{itemize}[leftmargin=*,noitemsep]
        \item Core optimization flow: Build hash table $\to$ Generate Rulebook $\to$ Gather input features $\to$ Execute GEMM $\to$ Scatter output results
        \item Quantization algorithm comparison: Systematically evaluated error distribution of three methods across different layers, selected optimal combination strategy
    \end{itemize}
    \item \textbf{Business Value}
    \begin{itemize}[leftmargin=*,noitemsep]
        \item Mass-produced 100+ MEC devices based on AMD Xilinx ZCU104
        \item Achieved 2M+ RMB in software-hardware revenue, served nearly 10 customers
        \item Mass production for Shaanxi Heavy Duty Automobile
    \end{itemize}
\end{itemize}

\textbf{[Project: NeurIPS 2025 CureBench Medical Agent System (CureAgent)]}

\vspace{4pt}
\begin{itemize}[leftmargin=*,itemsep=2pt,parsep=1pt]
    \item \textbf{Project Role:} NeurIPS 2025 Workshop Project Lead / Core Developer, Co-first Author
    \item \textbf{Project Background:} Participated in NeurIPS 2025 top-tier conference Workshop, built professional Agent system in biomedical field. Addressed the \textbf{Context Utilization Failure} problem in current clinical agents built on small LLMs (e.g., TxAgent), where models successfully retrieve biomedical evidence but fail to ground their diagnosis in that information. Proposed the Executor-Analyst Framework that decouples syntactic precision of tool execution from semantic robustness of clinical reasoning.
    \item \textbf{Technical Architecture:} CureBench + TxAgent RL Framework - ART Training
    \begin{itemize}[leftmargin=*,noitemsep]
        \item \textbf{Executor-Analyst Framework:} Modular architecture orchestrating specialized TxAgent Executors with long-context foundation model Analysts, mitigating reasoning deficits in monolithic models
        \item \textbf{Stratified Ensemble Strategy:} Significantly outperforms global pooling by preserving evidentiary diversity, effectively addressing the information bottleneck
        \item \textbf{Training-free Architectural Engineering:} Achieves state-of-the-art performance on CURE-Bench without expensive end-to-end finetuning
    \end{itemize}
    \item \textbf{Core Technical Breakthroughs:}
    \begin{itemize}[leftmargin=*,noitemsep]
        \item \textbf{Context-Performance Paradox:} Discovered that extending reasoning contexts beyond 12k tokens introduces noise that degrades accuracy
        \item \textbf{Curse of Dimensionality in Action Spaces:} Expanding toolsets necessitates hierarchical retrieval strategies
        \item Integrated biomedical tools (FDA, OpenTargets, PubMed) for agent tool-augmented reasoning
        \item Implemented Test-time Scaling technology to improve agent performance during testing phase
        \item Conducted agent training and optimization based on reinforcement learning framework (TxAgent RL Framework)
    \end{itemize}
    \item \textbf{Project Results:}
    \begin{itemize}[leftmargin=*,noitemsep]
        \item Achieved global 2nd place (Top 2) in CureBench International Agent Evaluation Competition
        \item Published paper to arXiv (arXiv:2512.05576), code released open-source
        \item Successfully built practical biomedical field agent system, providing scalable, agile foundation for next-generation trustworthy AI-driven therapeutics
    \end{itemize}
\end{itemize}

\textbf{Large Model Fine-tuning \& Deployment Projects}

\vspace{4pt}
\begin{itemize}[leftmargin=*,itemsep=2pt,parsep=1pt]
    \item Deployed Baichuan2 model on AWS as OpenAI-compatible format, achieved LoRA fine-tuning
    \item Used LLaMA-Factory on Alibaba Cloud platform for large model fine-tuning and optimization
    \item Deployed Start Code model to HuggingFace as API service
    \item Optimized 8B model performance to equivalent of 671B model level
\end{itemize}

\vspace{10pt}
\rule{\textwidth}{1pt}
\vspace{8pt}

\section*{Academic \& Competition Experience}

\begin{itemize}[leftmargin=*,itemsep=2pt,parsep=1pt]
    \item 2025: NeurIPS 2025 Agent Tool-Augmented Reasoning Workshop - CureBench International Agent Evaluation Competition Global 2nd Place (Top 2)
    \item Oct. 2021: CVPR Workshop - 3D Object Detection Algorithm (9th Place)
    \item Multiple Hackathon Awards:
    \begin{itemize}[leftmargin=*,itemsep=1pt]
        \item Oct. 2023: Baichuan Hackathon - US Traditional Chinese Medicine Customer Acquisition Agent Project (Special Award)
        \item Sep. 2023: Google I/O Hackathon - Stable Diffusion Computing Sharing (3rd Place)
        \item Jul. 2023: World AI Hackathon - Leadership Tracking \& Training (2nd Place)
        \item Sep. 2019: AngelHack Shanghai - Leadership Tracking \& Training (1st Place)
        \item Jul. 2017: Global AI Hackathon - Fake News Detection (Champion)
    \end{itemize}
\end{itemize}

\vspace{10pt}
\rule{\textwidth}{1pt}
\vspace{8pt}

\section*{Technical Community Involvement}

\begin{itemize}[leftmargin=*,itemsep=2pt,parsep=1pt]
    \item Google Machine Learning Developer Expert, conducted 4 technical lectures annually for 5 consecutive years, impacted 11,874 people
\end{itemize}

\end{document}

